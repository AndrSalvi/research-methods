\documentclass[10pt]{beamer}

\usetheme{metropolis}
\usepackage{appendixnumberbeamer}
\usepackage{libertine}
\usepackage[utf8]{inputenc}
\usepackage{csquotes}
\usepackage{url}
\urlstyle{same}

\usepackage{booktabs}
\usepackage[scale=2]{ccicons}

\usepackage{xspace}
\newcommand{\themename}{\textbf{\textsc{metropolis}}\xspace}

\title{Tutorial 01, Michaelmas Term}
\subtitle{Research Methods for Political Science (PO2600)}
\date{10 October 2017}
\author{Stefan Müller}
\institute{Trinity College Dublin}
% \titlegraphic{\hfill\includegraphics[height=1.5cm]{logo.pdf}}

\begin{document}

\maketitle

\begin{frame}{Table of contents}
  \setbeamertemplate{section in toc}[sections numbered]
  \tableofcontents%[hideallsubsections]
\end{frame}

\section{Tutorial Structure}
\begin{frame}{Tutorial Structure}

\begin{itemize}
\item Deepen and apply knowledge from the lectures
\item Learn how to use SPSS
\item Apply theories, concept and statistical methods to real-world data
\item Clarify questions, discuss homework
\item \textbf{But tutorials do not replace the lectures!}
\end{itemize}

\end{frame}

\begin{frame}{Grades}

Students taking the entire module:

\begin{enumerate}
\item 60\% of mark based on end-of-year exam (covers methods and statistics)
\item 2 homework assignments counting 4\% (1 during MT, 1 during HT)
\item 2 papers counting 10\% (one at the end of each term). Work will be done in pairs submitting joint papers.
\item 8 homework exercises (4 per term). Submit online via Turnitin \textit{before class}.
\end{enumerate}
\end{frame}

\begin{frame}{Grades}

Exchange students (one term only)
\begin{enumerate}
\item 1 homework assignment counting 12\%
\item 80\% of the mark based on two papers: a research proposal (30\%) and a final paper based on that proposal (50\%).
\item 8\% based on the 4 homework exercises to be submitted \textit{before} the tutorials.
\end{enumerate}
\end{frame}

\begin{frame}{Turnitin}

Separate Turnitin modules per term.

MT: Class ID: \textbf{16383023}; Password: \textbf{po3600}

HT: TBD

\end{frame}

\begin{frame}{Dates for Michaelmas Term}

\textbf{Homework}

Submit 4 homework exercises per term on Monday evening (11:59pm) preceding the tutorial session

\begin{itemize}
\item Week 4: HW 1 (next Monday!)
\item Week 6: HW 2
\item Week 9: HW 3
\item Week 11: HW 4
\end{itemize}

\textbf{Paper deadlines}

\begin{itemize}
\item Homework 1: 10 November 2017, 11:59pm
\item Research proposal (one-term students only!): 24 November, 11:59pm
\item Paper 1: 15 December 2017, 11:59pm
\end{itemize}


\end{frame}

\section{Support \& Additional Material}

\begin{frame}{Support}

\begin{itemize}
\item Constant feedback through short surveys% (content, teaching and tutorial style, too fast/slow?)
\item Notes, useful links and literature: \url{http://muellerstefan.net/po3600}
\item Questions: mullers@tcd.ie
\end{itemize}

\end{frame}

\section{How to Use SPSS}
\begin{frame}{SPSS}
\begin{itemize}
\item How to open (data) in SPSS?
\item How to work reproducibly in SPSS?
\end{itemize}
\end{frame}


\section{Discussion of Lecture Topics}
\begin{frame}{Central terms and definitions -- Lectures Week 01}

\begin{itemize}
\item Population
\item Sample
\item Random sample
\item Probability
\end{itemize}

\end{frame}

\begin{frame}{Central Terms}

\begin{itemize}
\item Concept
\item Theory
\item Deduction
\item Induction
\item Levels of measurement: nominal, ordinal, interval-ratio
\end{itemize}

\end{frame}


\end{document}